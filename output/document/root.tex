\documentclass[11pt,a4paper]{article}
\usepackage[T1]{fontenc}
\usepackage{isabelle,isabellesym}

% this should be the last package used
\usepackage{pdfsetup}

% urls in roman style, theory text in math-similar italics
\urlstyle{rm}
\isabellestyle{it}


\begin{document}

\title{Maximum Segment Sum}
\author{Nils Cremer}
\maketitle

\begin{abstract}
  The \emph{maximum segment sum} problem is to compute, given a list of numbers,
the largest of the sums of the contiguous segments of that list. It is also known
as the \emph{maximum sum subarray} problem and has been considered many times in the literature;
the Wikipedia article ``Maximum subarray problem''\cite{wiki} is a good starting point.

We assume that the elements of the list are not necessarily numbers but just elements
of some linearly ordered additive Abelian group.
\end{abstract}

\tableofcontents

% sane default for proof documents
\parindent 0pt\parskip 0.5ex

% generated text of all theories
%
\begin{isabellebody}%
\setisabellecontext{Maximum{\isacharunderscore}{\kern0pt}Segment{\isacharunderscore}{\kern0pt}Sum}%
%
\isadelimdocument
%
\endisadelimdocument
%
\isatagdocument
%
\isamarkupsection{Maximum Segment Sum%
}
\isamarkuptrue%
%
\endisatagdocument
{\isafolddocument}%
%
\isadelimdocument
%
\endisadelimdocument
%
\isadelimtheory
%
\endisadelimtheory
%
\isatagtheory
\isacommand{theory}\isamarkupfalse%
\ Maximum{\isacharunderscore}{\kern0pt}Segment{\isacharunderscore}{\kern0pt}Sum\isanewline
\ \ \isakeyword{imports}\ Main\isanewline
\isakeyword{begin}%
\endisatagtheory
{\isafoldtheory}%
%
\isadelimtheory
%
\endisadelimtheory
%
\begin{isamarkuptext}%
The \emph{maximum segment sum} problem is to compute, given a list of numbers,
the largest of the sums of the contiguous segments of that list. It is also known
as the \emph{maximum sum subarray} problem and has been considered many times in the literature;
the Wikipedia article ``Maximum subarray problem''
\url{https://en.wikipedia.org/wiki/Maximum_subarray_problem} is a good starting point.

We assume that the elements of the list are not necessarily numbers but just elements
of some linearly ordered group.%
\end{isamarkuptext}\isamarkuptrue%
\isacommand{class}\isamarkupfalse%
\ linordered{\isacharunderscore}{\kern0pt}group{\isacharunderscore}{\kern0pt}add\ {\isacharequal}{\kern0pt}\ linorder\ {\isacharplus}{\kern0pt}\ group{\isacharunderscore}{\kern0pt}add\ {\isacharplus}{\kern0pt}\isanewline
\isakeyword{assumes}\ add{\isacharunderscore}{\kern0pt}left{\isacharunderscore}{\kern0pt}mono{\isacharcolon}{\kern0pt}\ {\isachardoublequoteopen}a\ {\isasymle}\ b\ {\isasymLongrightarrow}\ c\ {\isacharplus}{\kern0pt}\ a\ {\isasymle}\ c\ {\isacharplus}{\kern0pt}\ b{\isachardoublequoteclose}\isanewline
\isakeyword{assumes}\ add{\isacharunderscore}{\kern0pt}right{\isacharunderscore}{\kern0pt}mono{\isacharcolon}{\kern0pt}\ {\isachardoublequoteopen}a\ {\isasymle}\ b\ {\isasymLongrightarrow}\ a\ {\isacharplus}{\kern0pt}\ c\ {\isasymle}\ b\ {\isacharplus}{\kern0pt}\ c{\isachardoublequoteclose}\isanewline
\isakeyword{begin}\isanewline
\isanewline
\isacommand{lemma}\isamarkupfalse%
\ max{\isacharunderscore}{\kern0pt}add{\isacharunderscore}{\kern0pt}distrib{\isacharunderscore}{\kern0pt}left{\isacharcolon}{\kern0pt}\ {\isachardoublequoteopen}max\ y\ z\ {\isacharplus}{\kern0pt}\ x\ {\isacharequal}{\kern0pt}\ max\ {\isacharparenleft}{\kern0pt}y{\isacharplus}{\kern0pt}x{\isacharparenright}{\kern0pt}\ {\isacharparenleft}{\kern0pt}z{\isacharplus}{\kern0pt}x{\isacharparenright}{\kern0pt}{\isachardoublequoteclose}\isanewline
%
\isadelimproof
%
\endisadelimproof
%
\isatagproof
\isacommand{by}\isamarkupfalse%
\ {\isacharparenleft}{\kern0pt}metis\ add{\isacharunderscore}{\kern0pt}right{\isacharunderscore}{\kern0pt}mono\ max{\isachardot}{\kern0pt}absorb{\isacharunderscore}{\kern0pt}iff{\isadigit{1}}\ max{\isacharunderscore}{\kern0pt}def{\isacharparenright}{\kern0pt}%
\endisatagproof
{\isafoldproof}%
%
\isadelimproof
\isanewline
%
\endisadelimproof
\isanewline
\isacommand{lemma}\isamarkupfalse%
\ max{\isacharunderscore}{\kern0pt}add{\isacharunderscore}{\kern0pt}distrib{\isacharunderscore}{\kern0pt}right{\isacharcolon}{\kern0pt}\ {\isachardoublequoteopen}x\ {\isacharplus}{\kern0pt}\ max\ y\ z\ {\isacharequal}{\kern0pt}\ max\ {\isacharparenleft}{\kern0pt}x{\isacharplus}{\kern0pt}y{\isacharparenright}{\kern0pt}\ {\isacharparenleft}{\kern0pt}x{\isacharplus}{\kern0pt}z{\isacharparenright}{\kern0pt}{\isachardoublequoteclose}\isanewline
%
\isadelimproof
%
\endisadelimproof
%
\isatagproof
\isacommand{by}\isamarkupfalse%
\ {\isacharparenleft}{\kern0pt}metis\ add{\isacharunderscore}{\kern0pt}left{\isacharunderscore}{\kern0pt}mono\ max{\isachardot}{\kern0pt}absorb{\isadigit{1}}\ max{\isachardot}{\kern0pt}cobounded{\isadigit{2}}\ max{\isacharunderscore}{\kern0pt}def{\isacharparenright}{\kern0pt}%
\endisatagproof
{\isafoldproof}%
%
\isadelimproof
%
\endisadelimproof
%
\isadelimdocument
%
\endisadelimdocument
%
\isatagdocument
%
\isamarkupsubsection{Naive Solution%
}
\isamarkuptrue%
%
\endisatagdocument
{\isafolddocument}%
%
\isadelimdocument
%
\endisadelimdocument
\isacommand{fun}\isamarkupfalse%
\ mss{\isacharunderscore}{\kern0pt}rec{\isacharunderscore}{\kern0pt}naive{\isacharunderscore}{\kern0pt}aux\ {\isacharcolon}{\kern0pt}{\isacharcolon}{\kern0pt}\ {\isachardoublequoteopen}{\isacharprime}{\kern0pt}a\ list\ {\isasymRightarrow}\ {\isacharprime}{\kern0pt}a{\isachardoublequoteclose}\ \isakeyword{where}\isanewline
\ \ {\isachardoublequoteopen}mss{\isacharunderscore}{\kern0pt}rec{\isacharunderscore}{\kern0pt}naive{\isacharunderscore}{\kern0pt}aux\ {\isacharbrackleft}{\kern0pt}{\isacharbrackright}{\kern0pt}\ {\isacharequal}{\kern0pt}\ {\isadigit{0}}{\isachardoublequoteclose}\isanewline
{\isacharbar}{\kern0pt}\ {\isachardoublequoteopen}mss{\isacharunderscore}{\kern0pt}rec{\isacharunderscore}{\kern0pt}naive{\isacharunderscore}{\kern0pt}aux\ {\isacharparenleft}{\kern0pt}x{\isacharhash}{\kern0pt}xs{\isacharparenright}{\kern0pt}\ {\isacharequal}{\kern0pt}\ max\ {\isadigit{0}}\ {\isacharparenleft}{\kern0pt}x\ {\isacharplus}{\kern0pt}\ mss{\isacharunderscore}{\kern0pt}rec{\isacharunderscore}{\kern0pt}naive{\isacharunderscore}{\kern0pt}aux\ xs{\isacharparenright}{\kern0pt}{\isachardoublequoteclose}\isanewline
\isanewline
\isacommand{fun}\isamarkupfalse%
\ mss{\isacharunderscore}{\kern0pt}rec{\isacharunderscore}{\kern0pt}naive\ {\isacharcolon}{\kern0pt}{\isacharcolon}{\kern0pt}\ {\isachardoublequoteopen}{\isacharprime}{\kern0pt}a\ list\ {\isasymRightarrow}\ {\isacharprime}{\kern0pt}a{\isachardoublequoteclose}\ \isakeyword{where}\isanewline
\ \ {\isachardoublequoteopen}mss{\isacharunderscore}{\kern0pt}rec{\isacharunderscore}{\kern0pt}naive\ {\isacharbrackleft}{\kern0pt}{\isacharbrackright}{\kern0pt}\ {\isacharequal}{\kern0pt}\ {\isadigit{0}}{\isachardoublequoteclose}\isanewline
{\isacharbar}{\kern0pt}\ {\isachardoublequoteopen}mss{\isacharunderscore}{\kern0pt}rec{\isacharunderscore}{\kern0pt}naive\ {\isacharparenleft}{\kern0pt}x{\isacharhash}{\kern0pt}xs{\isacharparenright}{\kern0pt}\ {\isacharequal}{\kern0pt}\ max\ {\isacharparenleft}{\kern0pt}mss{\isacharunderscore}{\kern0pt}rec{\isacharunderscore}{\kern0pt}naive{\isacharunderscore}{\kern0pt}aux\ {\isacharparenleft}{\kern0pt}x{\isacharhash}{\kern0pt}xs{\isacharparenright}{\kern0pt}{\isacharparenright}{\kern0pt}\ {\isacharparenleft}{\kern0pt}mss{\isacharunderscore}{\kern0pt}rec{\isacharunderscore}{\kern0pt}naive\ xs{\isacharparenright}{\kern0pt}{\isachardoublequoteclose}\isanewline
\isanewline
\isacommand{definition}\isamarkupfalse%
\ fronts\ {\isacharcolon}{\kern0pt}{\isacharcolon}{\kern0pt}\ {\isachardoublequoteopen}{\isacharprime}{\kern0pt}a\ list\ {\isasymRightarrow}\ {\isacharprime}{\kern0pt}a\ list\ set{\isachardoublequoteclose}\ \isakeyword{where}\isanewline
\ \ {\isachardoublequoteopen}fronts\ xs\ {\isacharequal}{\kern0pt}\ {\isacharbraceleft}{\kern0pt}as{\isachardot}{\kern0pt}\ {\isasymexists}bs{\isachardot}{\kern0pt}\ xs\ {\isacharequal}{\kern0pt}\ as\ {\isacharat}{\kern0pt}\ bs{\isacharbraceright}{\kern0pt}{\isachardoublequoteclose}\isanewline
\isanewline
\isacommand{definition}\isamarkupfalse%
\ {\isachardoublequoteopen}front{\isacharunderscore}{\kern0pt}sums\ xs\ {\isasymequiv}\ sum{\isacharunderscore}{\kern0pt}list\ {\isacharbackquote}{\kern0pt}\ fronts\ xs{\isachardoublequoteclose}\isanewline
\isanewline
\isacommand{lemma}\isamarkupfalse%
\ fronts{\isacharunderscore}{\kern0pt}cons{\isacharcolon}{\kern0pt}\ {\isachardoublequoteopen}fronts\ {\isacharparenleft}{\kern0pt}x{\isacharhash}{\kern0pt}xs{\isacharparenright}{\kern0pt}\ {\isacharequal}{\kern0pt}\ {\isacharparenleft}{\kern0pt}{\isacharparenleft}{\kern0pt}{\isacharhash}{\kern0pt}{\isacharparenright}{\kern0pt}\ x{\isacharparenright}{\kern0pt}\ {\isacharbackquote}{\kern0pt}\ fronts\ xs\ {\isasymunion}\ {\isacharbraceleft}{\kern0pt}{\isacharbrackleft}{\kern0pt}{\isacharbrackright}{\kern0pt}{\isacharbraceright}{\kern0pt}{\isachardoublequoteclose}\ {\isacharparenleft}{\kern0pt}\isakeyword{is}\ {\isachardoublequoteopen}{\isacharquery}{\kern0pt}l\ {\isacharequal}{\kern0pt}\ {\isacharquery}{\kern0pt}r{\isachardoublequoteclose}{\isacharparenright}{\kern0pt}\isanewline
%
\isadelimproof
%
\endisadelimproof
%
\isatagproof
\isacommand{proof}\isamarkupfalse%
\isanewline
\ \ \isacommand{show}\isamarkupfalse%
\ {\isachardoublequoteopen}{\isacharquery}{\kern0pt}l\ {\isasymsubseteq}\ {\isacharquery}{\kern0pt}r{\isachardoublequoteclose}\isanewline
\ \ \isacommand{proof}\isamarkupfalse%
\isanewline
\ \ \ \ \isacommand{fix}\isamarkupfalse%
\ as\ \isacommand{assume}\isamarkupfalse%
\ {\isachardoublequoteopen}as\ {\isasymin}\ {\isacharquery}{\kern0pt}l{\isachardoublequoteclose}\isanewline
\ \ \ \ \isacommand{then}\isamarkupfalse%
\ \isacommand{show}\isamarkupfalse%
\ {\isachardoublequoteopen}as\ {\isasymin}\ {\isacharquery}{\kern0pt}r{\isachardoublequoteclose}\ \isacommand{by}\isamarkupfalse%
\ {\isacharparenleft}{\kern0pt}cases\ as{\isacharparenright}{\kern0pt}\ {\isacharparenleft}{\kern0pt}auto\ simp{\isacharcolon}{\kern0pt}\ fronts{\isacharunderscore}{\kern0pt}def{\isacharparenright}{\kern0pt}\isanewline
\ \ \isacommand{qed}\isamarkupfalse%
\isanewline
\ \ \isacommand{show}\isamarkupfalse%
\ {\isachardoublequoteopen}{\isacharquery}{\kern0pt}r\ {\isasymsubseteq}\ {\isacharquery}{\kern0pt}l{\isachardoublequoteclose}\ \isacommand{unfolding}\isamarkupfalse%
\ fronts{\isacharunderscore}{\kern0pt}def\ \isacommand{by}\isamarkupfalse%
\ auto\isanewline
\isacommand{qed}\isamarkupfalse%
%
\endisatagproof
{\isafoldproof}%
%
\isadelimproof
\isanewline
%
\endisadelimproof
\isanewline
\isacommand{lemma}\isamarkupfalse%
\ front{\isacharunderscore}{\kern0pt}sums{\isacharunderscore}{\kern0pt}cons{\isacharcolon}{\kern0pt}\ {\isachardoublequoteopen}front{\isacharunderscore}{\kern0pt}sums\ {\isacharparenleft}{\kern0pt}x{\isacharhash}{\kern0pt}xs{\isacharparenright}{\kern0pt}\ {\isacharequal}{\kern0pt}\ {\isacharparenleft}{\kern0pt}{\isacharplus}{\kern0pt}{\isacharparenright}{\kern0pt}\ x\ {\isacharbackquote}{\kern0pt}\ front{\isacharunderscore}{\kern0pt}sums\ xs\ {\isasymunion}\ {\isacharbraceleft}{\kern0pt}{\isadigit{0}}{\isacharbraceright}{\kern0pt}{\isachardoublequoteclose}\isanewline
%
\isadelimproof
%
\endisadelimproof
%
\isatagproof
\isacommand{proof}\isamarkupfalse%
\ {\isacharminus}{\kern0pt}\isanewline
\ \ \isacommand{have}\isamarkupfalse%
\ {\isachardoublequoteopen}sum{\isacharunderscore}{\kern0pt}list\ {\isacharbackquote}{\kern0pt}\ {\isacharparenleft}{\kern0pt}{\isacharparenleft}{\kern0pt}{\isacharhash}{\kern0pt}{\isacharparenright}{\kern0pt}\ x{\isacharparenright}{\kern0pt}\ {\isacharbackquote}{\kern0pt}\ fronts\ xs\ {\isacharequal}{\kern0pt}\ {\isacharparenleft}{\kern0pt}{\isacharplus}{\kern0pt}{\isacharparenright}{\kern0pt}\ x\ {\isacharbackquote}{\kern0pt}\ front{\isacharunderscore}{\kern0pt}sums\ xs{\isachardoublequoteclose}\ \isacommand{unfolding}\isamarkupfalse%
\ front{\isacharunderscore}{\kern0pt}sums{\isacharunderscore}{\kern0pt}def\ \isacommand{by}\isamarkupfalse%
\ force\isanewline
\ \ \isacommand{then}\isamarkupfalse%
\ \isacommand{show}\isamarkupfalse%
\ {\isacharquery}{\kern0pt}thesis\ \isacommand{by}\isamarkupfalse%
\ {\isacharparenleft}{\kern0pt}simp\ add{\isacharcolon}{\kern0pt}\ front{\isacharunderscore}{\kern0pt}sums{\isacharunderscore}{\kern0pt}def\ fronts{\isacharunderscore}{\kern0pt}cons{\isacharparenright}{\kern0pt}\isanewline
\isacommand{qed}\isamarkupfalse%
%
\endisatagproof
{\isafoldproof}%
%
\isadelimproof
\isanewline
%
\endisadelimproof
\isanewline
\isacommand{lemma}\isamarkupfalse%
\ finite{\isacharunderscore}{\kern0pt}fronts{\isacharcolon}{\kern0pt}\ {\isachardoublequoteopen}finite\ {\isacharparenleft}{\kern0pt}fronts\ xs{\isacharparenright}{\kern0pt}{\isachardoublequoteclose}\isanewline
%
\isadelimproof
\ \ %
\endisadelimproof
%
\isatagproof
\isacommand{by}\isamarkupfalse%
\ {\isacharparenleft}{\kern0pt}induction\ xs{\isacharparenright}{\kern0pt}\ {\isacharparenleft}{\kern0pt}simp\ add{\isacharcolon}{\kern0pt}\ fronts{\isacharunderscore}{\kern0pt}def{\isacharcomma}{\kern0pt}\ simp\ add{\isacharcolon}{\kern0pt}\ fronts{\isacharunderscore}{\kern0pt}cons{\isacharparenright}{\kern0pt}%
\endisatagproof
{\isafoldproof}%
%
\isadelimproof
\isanewline
%
\endisadelimproof
\isanewline
\isacommand{lemma}\isamarkupfalse%
\ finite{\isacharunderscore}{\kern0pt}front{\isacharunderscore}{\kern0pt}sums{\isacharcolon}{\kern0pt}\ {\isachardoublequoteopen}finite\ {\isacharparenleft}{\kern0pt}front{\isacharunderscore}{\kern0pt}sums\ xs{\isacharparenright}{\kern0pt}{\isachardoublequoteclose}\isanewline
%
\isadelimproof
\ \ %
\endisadelimproof
%
\isatagproof
\isacommand{using}\isamarkupfalse%
\ front{\isacharunderscore}{\kern0pt}sums{\isacharunderscore}{\kern0pt}def\ finite{\isacharunderscore}{\kern0pt}fronts\ \isacommand{by}\isamarkupfalse%
\ simp%
\endisatagproof
{\isafoldproof}%
%
\isadelimproof
\isanewline
%
\endisadelimproof
\isanewline
\isacommand{lemma}\isamarkupfalse%
\ front{\isacharunderscore}{\kern0pt}sums{\isacharunderscore}{\kern0pt}not{\isacharunderscore}{\kern0pt}empty{\isacharcolon}{\kern0pt}\ {\isachardoublequoteopen}front{\isacharunderscore}{\kern0pt}sums\ xs\ {\isasymnoteq}\ {\isacharbraceleft}{\kern0pt}{\isacharbraceright}{\kern0pt}{\isachardoublequoteclose}\isanewline
%
\isadelimproof
\ \ %
\endisadelimproof
%
\isatagproof
\isacommand{unfolding}\isamarkupfalse%
\ front{\isacharunderscore}{\kern0pt}sums{\isacharunderscore}{\kern0pt}def\ fronts{\isacharunderscore}{\kern0pt}def\ \isacommand{using}\isamarkupfalse%
\ image{\isacharunderscore}{\kern0pt}iff\ \isacommand{by}\isamarkupfalse%
\ fastforce%
\endisatagproof
{\isafoldproof}%
%
\isadelimproof
\isanewline
%
\endisadelimproof
\isanewline
\isacommand{lemma}\isamarkupfalse%
\ max{\isacharunderscore}{\kern0pt}front{\isacharunderscore}{\kern0pt}sum{\isacharcolon}{\kern0pt}\ {\isachardoublequoteopen}Max\ {\isacharparenleft}{\kern0pt}front{\isacharunderscore}{\kern0pt}sums\ {\isacharparenleft}{\kern0pt}x{\isacharhash}{\kern0pt}xs{\isacharparenright}{\kern0pt}{\isacharparenright}{\kern0pt}\ {\isacharequal}{\kern0pt}\ max\ {\isadigit{0}}\ {\isacharparenleft}{\kern0pt}x\ {\isacharplus}{\kern0pt}\ Max\ {\isacharparenleft}{\kern0pt}front{\isacharunderscore}{\kern0pt}sums\ xs{\isacharparenright}{\kern0pt}{\isacharparenright}{\kern0pt}{\isachardoublequoteclose}\isanewline
%
\isadelimproof
%
\endisadelimproof
%
\isatagproof
\isacommand{using}\isamarkupfalse%
\ finite{\isacharunderscore}{\kern0pt}front{\isacharunderscore}{\kern0pt}sums\ front{\isacharunderscore}{\kern0pt}sums{\isacharunderscore}{\kern0pt}not{\isacharunderscore}{\kern0pt}empty\isanewline
\isacommand{by}\isamarkupfalse%
\ {\isacharparenleft}{\kern0pt}auto\ simp\ add{\isacharcolon}{\kern0pt}\ front{\isacharunderscore}{\kern0pt}sums{\isacharunderscore}{\kern0pt}cons\ hom{\isacharunderscore}{\kern0pt}Max{\isacharunderscore}{\kern0pt}commute\ max{\isacharunderscore}{\kern0pt}add{\isacharunderscore}{\kern0pt}distrib{\isacharunderscore}{\kern0pt}right{\isacharparenright}{\kern0pt}%
\endisatagproof
{\isafoldproof}%
%
\isadelimproof
\isanewline
%
\endisadelimproof
\isanewline
\isacommand{lemma}\isamarkupfalse%
\ mss{\isacharunderscore}{\kern0pt}rec{\isacharunderscore}{\kern0pt}naive{\isacharunderscore}{\kern0pt}aux{\isacharunderscore}{\kern0pt}front{\isacharunderscore}{\kern0pt}sums{\isacharcolon}{\kern0pt}\ {\isachardoublequoteopen}mss{\isacharunderscore}{\kern0pt}rec{\isacharunderscore}{\kern0pt}naive{\isacharunderscore}{\kern0pt}aux\ xs\ {\isacharequal}{\kern0pt}\ Max\ {\isacharparenleft}{\kern0pt}front{\isacharunderscore}{\kern0pt}sums\ xs{\isacharparenright}{\kern0pt}{\isachardoublequoteclose}\isanewline
%
\isadelimproof
%
\endisadelimproof
%
\isatagproof
\isacommand{by}\isamarkupfalse%
\ {\isacharparenleft}{\kern0pt}induction\ xs{\isacharparenright}{\kern0pt}\ {\isacharparenleft}{\kern0pt}simp\ add{\isacharcolon}{\kern0pt}\ front{\isacharunderscore}{\kern0pt}sums{\isacharunderscore}{\kern0pt}def\ fronts{\isacharunderscore}{\kern0pt}def{\isacharcomma}{\kern0pt}\ auto\ simp{\isacharcolon}{\kern0pt}\ max{\isacharunderscore}{\kern0pt}front{\isacharunderscore}{\kern0pt}sum{\isacharparenright}{\kern0pt}%
\endisatagproof
{\isafoldproof}%
%
\isadelimproof
\isanewline
%
\endisadelimproof
\isanewline
\isacommand{lemma}\isamarkupfalse%
\ front{\isacharunderscore}{\kern0pt}sums{\isacharcolon}{\kern0pt}\ {\isachardoublequoteopen}front{\isacharunderscore}{\kern0pt}sums\ xs\ {\isacharequal}{\kern0pt}\ {\isacharbraceleft}{\kern0pt}s{\isachardot}{\kern0pt}\ {\isasymexists}as\ bs{\isachardot}{\kern0pt}\ xs\ {\isacharequal}{\kern0pt}\ as\ {\isacharat}{\kern0pt}\ bs\ {\isasymand}\ s\ {\isacharequal}{\kern0pt}\ sum{\isacharunderscore}{\kern0pt}list\ as{\isacharbraceright}{\kern0pt}{\isachardoublequoteclose}\isanewline
%
\isadelimproof
%
\endisadelimproof
%
\isatagproof
\isacommand{unfolding}\isamarkupfalse%
\ front{\isacharunderscore}{\kern0pt}sums{\isacharunderscore}{\kern0pt}def\ fronts{\isacharunderscore}{\kern0pt}def\ \isacommand{by}\isamarkupfalse%
\ auto%
\endisatagproof
{\isafoldproof}%
%
\isadelimproof
\isanewline
%
\endisadelimproof
\isanewline
\isacommand{lemma}\isamarkupfalse%
\ mss{\isacharunderscore}{\kern0pt}rec{\isacharunderscore}{\kern0pt}naive{\isacharunderscore}{\kern0pt}aux{\isacharcolon}{\kern0pt}\ {\isachardoublequoteopen}mss{\isacharunderscore}{\kern0pt}rec{\isacharunderscore}{\kern0pt}naive{\isacharunderscore}{\kern0pt}aux\ xs\ {\isacharequal}{\kern0pt}\ Max\ {\isacharbraceleft}{\kern0pt}s{\isachardot}{\kern0pt}\ {\isasymexists}as\ bs{\isachardot}{\kern0pt}\ xs\ {\isacharequal}{\kern0pt}\ as\ {\isacharat}{\kern0pt}\ bs\ {\isasymand}\ s\ {\isacharequal}{\kern0pt}\ sum{\isacharunderscore}{\kern0pt}list\ as{\isacharbraceright}{\kern0pt}{\isachardoublequoteclose}\isanewline
%
\isadelimproof
%
\endisadelimproof
%
\isatagproof
\isacommand{using}\isamarkupfalse%
\ front{\isacharunderscore}{\kern0pt}sums\ mss{\isacharunderscore}{\kern0pt}rec{\isacharunderscore}{\kern0pt}naive{\isacharunderscore}{\kern0pt}aux{\isacharunderscore}{\kern0pt}front{\isacharunderscore}{\kern0pt}sums\ \isacommand{by}\isamarkupfalse%
\ simp%
\endisatagproof
{\isafoldproof}%
%
\isadelimproof
\isanewline
%
\endisadelimproof
\ \ \isanewline
\isanewline
\isacommand{definition}\isamarkupfalse%
\ mids\ {\isacharcolon}{\kern0pt}{\isacharcolon}{\kern0pt}\ {\isachardoublequoteopen}{\isacharprime}{\kern0pt}a\ list\ {\isasymRightarrow}\ {\isacharprime}{\kern0pt}a\ list\ set{\isachardoublequoteclose}\ \isakeyword{where}\isanewline
\ \ {\isachardoublequoteopen}mids\ xs\ {\isasymequiv}\ {\isacharbraceleft}{\kern0pt}bs{\isachardot}{\kern0pt}\ {\isasymexists}as\ cs{\isachardot}{\kern0pt}\ xs\ {\isacharequal}{\kern0pt}\ as\ {\isacharat}{\kern0pt}\ bs\ {\isacharat}{\kern0pt}\ cs{\isacharbraceright}{\kern0pt}{\isachardoublequoteclose}\isanewline
\isanewline
\isacommand{definition}\isamarkupfalse%
\ {\isachardoublequoteopen}mid{\isacharunderscore}{\kern0pt}sums\ xs\ {\isasymequiv}\ sum{\isacharunderscore}{\kern0pt}list\ {\isacharbackquote}{\kern0pt}\ mids\ xs{\isachardoublequoteclose}\isanewline
\isanewline
\isacommand{lemma}\isamarkupfalse%
\ fronts{\isacharunderscore}{\kern0pt}mids{\isacharcolon}{\kern0pt}\ {\isachardoublequoteopen}bs\ {\isasymin}\ fronts\ xs\ {\isasymLongrightarrow}\ bs\ {\isasymin}\ mids\ xs{\isachardoublequoteclose}\isanewline
%
\isadelimproof
%
\endisadelimproof
%
\isatagproof
\isacommand{unfolding}\isamarkupfalse%
\ fronts{\isacharunderscore}{\kern0pt}def\ mids{\isacharunderscore}{\kern0pt}def\ \isacommand{by}\isamarkupfalse%
\ auto%
\endisatagproof
{\isafoldproof}%
%
\isadelimproof
\isanewline
%
\endisadelimproof
\isanewline
\isacommand{lemma}\isamarkupfalse%
\ mids{\isacharunderscore}{\kern0pt}mids{\isacharunderscore}{\kern0pt}cons{\isacharcolon}{\kern0pt}\ {\isachardoublequoteopen}bs\ {\isasymin}\ mids\ xs\ {\isasymLongrightarrow}\ bs\ {\isasymin}\ mids\ {\isacharparenleft}{\kern0pt}x{\isacharhash}{\kern0pt}xs{\isacharparenright}{\kern0pt}{\isachardoublequoteclose}\isanewline
%
\isadelimproof
%
\endisadelimproof
%
\isatagproof
\isacommand{proof}\isamarkupfalse%
{\isacharminus}{\kern0pt}\isanewline
\ \ \isacommand{fix}\isamarkupfalse%
\ bs\ \isacommand{assume}\isamarkupfalse%
\ {\isachardoublequoteopen}bs\ {\isasymin}\ mids\ xs{\isachardoublequoteclose}\isanewline
\ \ \isacommand{then}\isamarkupfalse%
\ \isacommand{obtain}\isamarkupfalse%
\ as\ cs\ \isakeyword{where}\ {\isachardoublequoteopen}xs\ {\isacharequal}{\kern0pt}\ as\ {\isacharat}{\kern0pt}\ bs\ {\isacharat}{\kern0pt}\ cs{\isachardoublequoteclose}\ \isacommand{unfolding}\isamarkupfalse%
\ mids{\isacharunderscore}{\kern0pt}def\ \isacommand{by}\isamarkupfalse%
\ blast\isanewline
\ \ \isacommand{then}\isamarkupfalse%
\ \isacommand{have}\isamarkupfalse%
\ {\isachardoublequoteopen}x\ {\isacharhash}{\kern0pt}\ xs\ {\isacharequal}{\kern0pt}\ {\isacharparenleft}{\kern0pt}x{\isacharhash}{\kern0pt}as{\isacharparenright}{\kern0pt}\ {\isacharat}{\kern0pt}\ bs\ {\isacharat}{\kern0pt}\ cs{\isachardoublequoteclose}\ \isacommand{by}\isamarkupfalse%
\ simp\isanewline
\ \ \isacommand{then}\isamarkupfalse%
\ \isacommand{show}\isamarkupfalse%
\ {\isachardoublequoteopen}bs\ {\isasymin}\ mids\ {\isacharparenleft}{\kern0pt}x{\isacharhash}{\kern0pt}xs{\isacharparenright}{\kern0pt}{\isachardoublequoteclose}\ \isacommand{unfolding}\isamarkupfalse%
\ mids{\isacharunderscore}{\kern0pt}def\ \isacommand{by}\isamarkupfalse%
\ blast\isanewline
\isacommand{qed}\isamarkupfalse%
%
\endisatagproof
{\isafoldproof}%
%
\isadelimproof
\isanewline
%
\endisadelimproof
\isanewline
\isacommand{lemma}\isamarkupfalse%
\ mids{\isacharunderscore}{\kern0pt}cons{\isacharcolon}{\kern0pt}\ {\isachardoublequoteopen}mids\ {\isacharparenleft}{\kern0pt}x{\isacharhash}{\kern0pt}xs{\isacharparenright}{\kern0pt}\ {\isacharequal}{\kern0pt}\ fronts\ {\isacharparenleft}{\kern0pt}x{\isacharhash}{\kern0pt}xs{\isacharparenright}{\kern0pt}\ {\isasymunion}\ mids\ xs{\isachardoublequoteclose}\ {\isacharparenleft}{\kern0pt}\isakeyword{is}\ {\isachardoublequoteopen}{\isacharquery}{\kern0pt}l\ {\isacharequal}{\kern0pt}\ {\isacharquery}{\kern0pt}r{\isachardoublequoteclose}{\isacharparenright}{\kern0pt}\isanewline
%
\isadelimproof
%
\endisadelimproof
%
\isatagproof
\isacommand{proof}\isamarkupfalse%
\isanewline
\ \ \isacommand{show}\isamarkupfalse%
\ {\isachardoublequoteopen}{\isacharquery}{\kern0pt}l\ {\isasymsubseteq}\ {\isacharquery}{\kern0pt}r{\isachardoublequoteclose}\isanewline
\ \ \isacommand{proof}\isamarkupfalse%
\isanewline
\ \ \ \ \isacommand{fix}\isamarkupfalse%
\ bs\ \isacommand{assume}\isamarkupfalse%
\ {\isachardoublequoteopen}bs\ {\isasymin}\ {\isacharquery}{\kern0pt}l{\isachardoublequoteclose}\isanewline
\ \ \ \ \isacommand{then}\isamarkupfalse%
\ \isacommand{obtain}\isamarkupfalse%
\ as\ cs\ \isakeyword{where}\ as{\isacharunderscore}{\kern0pt}bs{\isacharunderscore}{\kern0pt}cs{\isacharcolon}{\kern0pt}\ {\isachardoublequoteopen}{\isacharparenleft}{\kern0pt}x{\isacharhash}{\kern0pt}xs{\isacharparenright}{\kern0pt}\ {\isacharequal}{\kern0pt}\ as\ {\isacharat}{\kern0pt}\ bs\ {\isacharat}{\kern0pt}\ cs{\isachardoublequoteclose}\ \isacommand{unfolding}\isamarkupfalse%
\ mids{\isacharunderscore}{\kern0pt}def\ \isacommand{by}\isamarkupfalse%
\ blast\isanewline
\ \ \ \ \isacommand{then}\isamarkupfalse%
\ \isacommand{show}\isamarkupfalse%
\ {\isachardoublequoteopen}bs\ {\isasymin}\ {\isacharquery}{\kern0pt}r{\isachardoublequoteclose}\isanewline
\ \ \ \ \isacommand{proof}\isamarkupfalse%
\ {\isacharparenleft}{\kern0pt}cases\ as{\isacharparenright}{\kern0pt}\isanewline
\ \ \ \ \ \ \isacommand{case}\isamarkupfalse%
\ Nil\isanewline
\ \ \ \ \ \ \isacommand{then}\isamarkupfalse%
\ \isacommand{have}\isamarkupfalse%
\ {\isachardoublequoteopen}bs\ {\isasymin}\ fronts\ {\isacharparenleft}{\kern0pt}x{\isacharhash}{\kern0pt}xs{\isacharparenright}{\kern0pt}{\isachardoublequoteclose}\ \isacommand{by}\isamarkupfalse%
\ {\isacharparenleft}{\kern0pt}simp\ add{\isacharcolon}{\kern0pt}\ fronts{\isacharunderscore}{\kern0pt}def\ as{\isacharunderscore}{\kern0pt}bs{\isacharunderscore}{\kern0pt}cs{\isacharparenright}{\kern0pt}\isanewline
\ \ \ \ \ \ \isacommand{then}\isamarkupfalse%
\ \isacommand{show}\isamarkupfalse%
\ {\isacharquery}{\kern0pt}thesis\ \isacommand{by}\isamarkupfalse%
\ simp\isanewline
\ \ \ \ \isacommand{next}\isamarkupfalse%
\isanewline
\ \ \ \ \ \ \isacommand{case}\isamarkupfalse%
\ {\isacharparenleft}{\kern0pt}Cons\ a\ as{\isacharprime}{\kern0pt}{\isacharparenright}{\kern0pt}\isanewline
\ \ \ \ \ \ \isacommand{then}\isamarkupfalse%
\ \isacommand{have}\isamarkupfalse%
\ {\isachardoublequoteopen}xs\ {\isacharequal}{\kern0pt}\ as{\isacharprime}{\kern0pt}\ {\isacharat}{\kern0pt}\ bs\ {\isacharat}{\kern0pt}\ cs{\isachardoublequoteclose}\ \isacommand{using}\isamarkupfalse%
\ as{\isacharunderscore}{\kern0pt}bs{\isacharunderscore}{\kern0pt}cs\ \isacommand{by}\isamarkupfalse%
\ simp\isanewline
\ \ \ \ \ \ \isacommand{then}\isamarkupfalse%
\ \isacommand{show}\isamarkupfalse%
\ {\isacharquery}{\kern0pt}thesis\ \isacommand{unfolding}\isamarkupfalse%
\ mids{\isacharunderscore}{\kern0pt}def\ \isacommand{by}\isamarkupfalse%
\ auto\isanewline
\ \ \ \ \isacommand{qed}\isamarkupfalse%
\isanewline
\ \ \isacommand{qed}\isamarkupfalse%
\isanewline
\ \ \isacommand{show}\isamarkupfalse%
\ {\isachardoublequoteopen}{\isacharquery}{\kern0pt}r\ {\isasymsubseteq}\ {\isacharquery}{\kern0pt}l{\isachardoublequoteclose}\ \isacommand{using}\isamarkupfalse%
\ fronts{\isacharunderscore}{\kern0pt}mids\ mids{\isacharunderscore}{\kern0pt}mids{\isacharunderscore}{\kern0pt}cons\ \isacommand{by}\isamarkupfalse%
\ auto\isanewline
\isacommand{qed}\isamarkupfalse%
%
\endisatagproof
{\isafoldproof}%
%
\isadelimproof
\isanewline
%
\endisadelimproof
\isanewline
\isacommand{lemma}\isamarkupfalse%
\ mid{\isacharunderscore}{\kern0pt}sums{\isacharunderscore}{\kern0pt}cons{\isacharcolon}{\kern0pt}\ {\isachardoublequoteopen}mid{\isacharunderscore}{\kern0pt}sums\ {\isacharparenleft}{\kern0pt}x{\isacharhash}{\kern0pt}xs{\isacharparenright}{\kern0pt}\ {\isacharequal}{\kern0pt}\ front{\isacharunderscore}{\kern0pt}sums\ {\isacharparenleft}{\kern0pt}x{\isacharhash}{\kern0pt}xs{\isacharparenright}{\kern0pt}\ {\isasymunion}\ mid{\isacharunderscore}{\kern0pt}sums\ xs{\isachardoublequoteclose}\isanewline
%
\isadelimproof
\ \ %
\endisadelimproof
%
\isatagproof
\isacommand{unfolding}\isamarkupfalse%
\ mid{\isacharunderscore}{\kern0pt}sums{\isacharunderscore}{\kern0pt}def\ \isacommand{by}\isamarkupfalse%
\ {\isacharparenleft}{\kern0pt}auto\ simp{\isacharcolon}{\kern0pt}\ mids{\isacharunderscore}{\kern0pt}cons\ front{\isacharunderscore}{\kern0pt}sums{\isacharunderscore}{\kern0pt}def{\isacharparenright}{\kern0pt}%
\endisatagproof
{\isafoldproof}%
%
\isadelimproof
\isanewline
%
\endisadelimproof
\isanewline
\isacommand{lemma}\isamarkupfalse%
\ finite{\isacharunderscore}{\kern0pt}mids{\isacharcolon}{\kern0pt}\ {\isachardoublequoteopen}finite\ {\isacharparenleft}{\kern0pt}mids\ xs{\isacharparenright}{\kern0pt}{\isachardoublequoteclose}\isanewline
%
\isadelimproof
\ \ %
\endisadelimproof
%
\isatagproof
\isacommand{by}\isamarkupfalse%
\ {\isacharparenleft}{\kern0pt}induction\ xs{\isacharparenright}{\kern0pt}\ {\isacharparenleft}{\kern0pt}simp\ add{\isacharcolon}{\kern0pt}\ mids{\isacharunderscore}{\kern0pt}def{\isacharcomma}{\kern0pt}\ simp\ add{\isacharcolon}{\kern0pt}\ mids{\isacharunderscore}{\kern0pt}cons\ finite{\isacharunderscore}{\kern0pt}fronts{\isacharparenright}{\kern0pt}%
\endisatagproof
{\isafoldproof}%
%
\isadelimproof
\isanewline
%
\endisadelimproof
\isanewline
\isacommand{lemma}\isamarkupfalse%
\ finite{\isacharunderscore}{\kern0pt}mid{\isacharunderscore}{\kern0pt}sums{\isacharcolon}{\kern0pt}\ {\isachardoublequoteopen}finite\ {\isacharparenleft}{\kern0pt}mid{\isacharunderscore}{\kern0pt}sums\ xs{\isacharparenright}{\kern0pt}{\isachardoublequoteclose}\isanewline
%
\isadelimproof
\ \ %
\endisadelimproof
%
\isatagproof
\isacommand{by}\isamarkupfalse%
\ {\isacharparenleft}{\kern0pt}simp\ add{\isacharcolon}{\kern0pt}\ mid{\isacharunderscore}{\kern0pt}sums{\isacharunderscore}{\kern0pt}def\ finite{\isacharunderscore}{\kern0pt}mids{\isacharparenright}{\kern0pt}%
\endisatagproof
{\isafoldproof}%
%
\isadelimproof
\isanewline
%
\endisadelimproof
\isanewline
\isacommand{lemma}\isamarkupfalse%
\ mid{\isacharunderscore}{\kern0pt}sums{\isacharunderscore}{\kern0pt}not{\isacharunderscore}{\kern0pt}empty{\isacharcolon}{\kern0pt}\ {\isachardoublequoteopen}mid{\isacharunderscore}{\kern0pt}sums\ xs\ {\isasymnoteq}\ {\isacharbraceleft}{\kern0pt}{\isacharbraceright}{\kern0pt}{\isachardoublequoteclose}\isanewline
%
\isadelimproof
\ \ %
\endisadelimproof
%
\isatagproof
\isacommand{unfolding}\isamarkupfalse%
\ mid{\isacharunderscore}{\kern0pt}sums{\isacharunderscore}{\kern0pt}def\ mids{\isacharunderscore}{\kern0pt}def\ \isacommand{by}\isamarkupfalse%
\ blast%
\endisatagproof
{\isafoldproof}%
%
\isadelimproof
\isanewline
%
\endisadelimproof
\isanewline
\isacommand{lemma}\isamarkupfalse%
\ max{\isacharunderscore}{\kern0pt}mid{\isacharunderscore}{\kern0pt}sums{\isacharunderscore}{\kern0pt}cons{\isacharcolon}{\kern0pt}\ {\isachardoublequoteopen}Max\ {\isacharparenleft}{\kern0pt}mid{\isacharunderscore}{\kern0pt}sums\ {\isacharparenleft}{\kern0pt}x{\isacharhash}{\kern0pt}xs{\isacharparenright}{\kern0pt}{\isacharparenright}{\kern0pt}\ {\isacharequal}{\kern0pt}\ max\ {\isacharparenleft}{\kern0pt}Max\ {\isacharparenleft}{\kern0pt}front{\isacharunderscore}{\kern0pt}sums\ {\isacharparenleft}{\kern0pt}x{\isacharhash}{\kern0pt}xs{\isacharparenright}{\kern0pt}{\isacharparenright}{\kern0pt}{\isacharparenright}{\kern0pt}\ {\isacharparenleft}{\kern0pt}Max\ {\isacharparenleft}{\kern0pt}mid{\isacharunderscore}{\kern0pt}sums\ xs{\isacharparenright}{\kern0pt}{\isacharparenright}{\kern0pt}{\isachardoublequoteclose}\isanewline
%
\isadelimproof
\ \ %
\endisadelimproof
%
\isatagproof
\isacommand{by}\isamarkupfalse%
\ {\isacharparenleft}{\kern0pt}auto\ simp{\isacharcolon}{\kern0pt}\ mid{\isacharunderscore}{\kern0pt}sums{\isacharunderscore}{\kern0pt}cons\ Max{\isacharunderscore}{\kern0pt}Un\ finite{\isacharunderscore}{\kern0pt}front{\isacharunderscore}{\kern0pt}sums\ finite{\isacharunderscore}{\kern0pt}mid{\isacharunderscore}{\kern0pt}sums\ front{\isacharunderscore}{\kern0pt}sums{\isacharunderscore}{\kern0pt}not{\isacharunderscore}{\kern0pt}empty\ mid{\isacharunderscore}{\kern0pt}sums{\isacharunderscore}{\kern0pt}not{\isacharunderscore}{\kern0pt}empty{\isacharparenright}{\kern0pt}%
\endisatagproof
{\isafoldproof}%
%
\isadelimproof
\isanewline
%
\endisadelimproof
\isanewline
\isacommand{lemma}\isamarkupfalse%
\ mss{\isacharunderscore}{\kern0pt}rec{\isacharunderscore}{\kern0pt}naive{\isacharunderscore}{\kern0pt}max{\isacharunderscore}{\kern0pt}mid{\isacharunderscore}{\kern0pt}sum{\isacharcolon}{\kern0pt}\ {\isachardoublequoteopen}mss{\isacharunderscore}{\kern0pt}rec{\isacharunderscore}{\kern0pt}naive\ xs\ {\isacharequal}{\kern0pt}\ Max\ {\isacharparenleft}{\kern0pt}mid{\isacharunderscore}{\kern0pt}sums\ xs{\isacharparenright}{\kern0pt}{\isachardoublequoteclose}\isanewline
%
\isadelimproof
\ \ %
\endisadelimproof
%
\isatagproof
\isacommand{by}\isamarkupfalse%
\ {\isacharparenleft}{\kern0pt}induction\ xs{\isacharparenright}{\kern0pt}\ {\isacharparenleft}{\kern0pt}simp\ add{\isacharcolon}{\kern0pt}\ mid{\isacharunderscore}{\kern0pt}sums{\isacharunderscore}{\kern0pt}def\ mids{\isacharunderscore}{\kern0pt}def{\isacharcomma}{\kern0pt}\ auto\ simp{\isacharcolon}{\kern0pt}\ max{\isacharunderscore}{\kern0pt}mid{\isacharunderscore}{\kern0pt}sums{\isacharunderscore}{\kern0pt}cons\ mss{\isacharunderscore}{\kern0pt}rec{\isacharunderscore}{\kern0pt}naive{\isacharunderscore}{\kern0pt}aux\ front{\isacharunderscore}{\kern0pt}sums{\isacharparenright}{\kern0pt}%
\endisatagproof
{\isafoldproof}%
%
\isadelimproof
\isanewline
%
\endisadelimproof
\isanewline
\isacommand{lemma}\isamarkupfalse%
\ mid{\isacharunderscore}{\kern0pt}sums{\isacharcolon}{\kern0pt}\ {\isachardoublequoteopen}mid{\isacharunderscore}{\kern0pt}sums\ xs\ {\isacharequal}{\kern0pt}\ {\isacharbraceleft}{\kern0pt}s{\isachardot}{\kern0pt}\ {\isasymexists}as\ bs\ cs{\isachardot}{\kern0pt}\ xs\ {\isacharequal}{\kern0pt}\ as\ {\isacharat}{\kern0pt}\ bs\ {\isacharat}{\kern0pt}\ cs\ {\isasymand}\ s\ {\isacharequal}{\kern0pt}\ sum{\isacharunderscore}{\kern0pt}list\ bs{\isacharbraceright}{\kern0pt}{\isachardoublequoteclose}\isanewline
%
\isadelimproof
\ \ %
\endisadelimproof
%
\isatagproof
\isacommand{by}\isamarkupfalse%
\ {\isacharparenleft}{\kern0pt}auto\ simp{\isacharcolon}{\kern0pt}\ mid{\isacharunderscore}{\kern0pt}sums{\isacharunderscore}{\kern0pt}def\ mids{\isacharunderscore}{\kern0pt}def{\isacharparenright}{\kern0pt}%
\endisatagproof
{\isafoldproof}%
%
\isadelimproof
\isanewline
%
\endisadelimproof
\isanewline
\isacommand{theorem}\isamarkupfalse%
\ mss{\isacharunderscore}{\kern0pt}rec{\isacharunderscore}{\kern0pt}naive{\isacharcolon}{\kern0pt}\ {\isachardoublequoteopen}mss{\isacharunderscore}{\kern0pt}rec{\isacharunderscore}{\kern0pt}naive\ xs\ {\isacharequal}{\kern0pt}\ Max\ {\isacharbraceleft}{\kern0pt}s{\isachardot}{\kern0pt}\ {\isasymexists}as\ bs\ cs{\isachardot}{\kern0pt}\ xs\ {\isacharequal}{\kern0pt}\ as\ {\isacharat}{\kern0pt}\ bs\ {\isacharat}{\kern0pt}\ cs\ {\isasymand}\ s\ {\isacharequal}{\kern0pt}\ sum{\isacharunderscore}{\kern0pt}list\ bs{\isacharbraceright}{\kern0pt}{\isachardoublequoteclose}\isanewline
%
\isadelimproof
\ \ %
\endisadelimproof
%
\isatagproof
\isacommand{unfolding}\isamarkupfalse%
\ mss{\isacharunderscore}{\kern0pt}rec{\isacharunderscore}{\kern0pt}naive{\isacharunderscore}{\kern0pt}max{\isacharunderscore}{\kern0pt}mid{\isacharunderscore}{\kern0pt}sum\ mid{\isacharunderscore}{\kern0pt}sums\ \isacommand{by}\isamarkupfalse%
\ simp%
\endisatagproof
{\isafoldproof}%
%
\isadelimproof
%
\endisadelimproof
%
\isadelimdocument
%
\endisadelimdocument
%
\isatagdocument
%
\isamarkupsubsection{Kadane's Algorithms%
}
\isamarkuptrue%
%
\endisatagdocument
{\isafolddocument}%
%
\isadelimdocument
%
\endisadelimdocument
\isacommand{fun}\isamarkupfalse%
\ kadane\ {\isacharcolon}{\kern0pt}{\isacharcolon}{\kern0pt}\ {\isachardoublequoteopen}{\isacharprime}{\kern0pt}a\ list\ {\isasymRightarrow}\ {\isacharprime}{\kern0pt}a\ {\isasymRightarrow}\ {\isacharprime}{\kern0pt}a\ {\isasymRightarrow}\ {\isacharprime}{\kern0pt}a{\isachardoublequoteclose}\ \isakeyword{where}\isanewline
\ \ {\isachardoublequoteopen}kadane\ {\isacharbrackleft}{\kern0pt}{\isacharbrackright}{\kern0pt}\ cur\ m\ {\isacharequal}{\kern0pt}\ m{\isachardoublequoteclose}\isanewline
{\isacharbar}{\kern0pt}\ {\isachardoublequoteopen}kadane\ {\isacharparenleft}{\kern0pt}x{\isacharhash}{\kern0pt}xs{\isacharparenright}{\kern0pt}\ cur\ m\ {\isacharequal}{\kern0pt}\isanewline
\ \ \ \ {\isacharparenleft}{\kern0pt}let\ cur{\isacharprime}{\kern0pt}\ {\isacharequal}{\kern0pt}\ max\ {\isacharparenleft}{\kern0pt}cur\ {\isacharplus}{\kern0pt}\ x{\isacharparenright}{\kern0pt}\ x\ in\isanewline
\ \ \ \ \ \ kadane\ xs\ cur{\isacharprime}{\kern0pt}\ {\isacharparenleft}{\kern0pt}max\ m\ cur{\isacharprime}{\kern0pt}{\isacharparenright}{\kern0pt}{\isacharparenright}{\kern0pt}{\isachardoublequoteclose}\isanewline
\isanewline
\isacommand{definition}\isamarkupfalse%
\ {\isachardoublequoteopen}mss{\isacharunderscore}{\kern0pt}kadane\ xs\ {\isasymequiv}\ kadane\ xs\ {\isadigit{0}}\ {\isadigit{0}}{\isachardoublequoteclose}\isanewline
\isanewline
\isacommand{lemma}\isamarkupfalse%
\ Max{\isacharunderscore}{\kern0pt}front{\isacharunderscore}{\kern0pt}sums{\isacharunderscore}{\kern0pt}geq{\isacharunderscore}{\kern0pt}{\isadigit{0}}{\isacharcolon}{\kern0pt}\ {\isachardoublequoteopen}Max\ {\isacharparenleft}{\kern0pt}front{\isacharunderscore}{\kern0pt}sums\ xs{\isacharparenright}{\kern0pt}\ {\isasymge}\ {\isadigit{0}}{\isachardoublequoteclose}\isanewline
%
\isadelimproof
%
\endisadelimproof
%
\isatagproof
\isacommand{proof}\isamarkupfalse%
{\isacharminus}{\kern0pt}\isanewline
\ \ \isacommand{have}\isamarkupfalse%
\ {\isachardoublequoteopen}{\isacharbrackleft}{\kern0pt}{\isacharbrackright}{\kern0pt}\ {\isasymin}\ fronts\ xs{\isachardoublequoteclose}\ \isacommand{unfolding}\isamarkupfalse%
\ fronts{\isacharunderscore}{\kern0pt}def\ \isacommand{by}\isamarkupfalse%
\ blast\isanewline
\ \ \isacommand{then}\isamarkupfalse%
\ \isacommand{have}\isamarkupfalse%
\ {\isachardoublequoteopen}{\isadigit{0}}\ {\isasymin}\ front{\isacharunderscore}{\kern0pt}sums\ xs{\isachardoublequoteclose}\ \isacommand{unfolding}\isamarkupfalse%
\ front{\isacharunderscore}{\kern0pt}sums{\isacharunderscore}{\kern0pt}def\ \isacommand{by}\isamarkupfalse%
\ force\isanewline
\ \ \isacommand{then}\isamarkupfalse%
\ \isacommand{show}\isamarkupfalse%
\ {\isacharquery}{\kern0pt}thesis\ \isacommand{using}\isamarkupfalse%
\ finite{\isacharunderscore}{\kern0pt}front{\isacharunderscore}{\kern0pt}sums\ Max{\isacharunderscore}{\kern0pt}ge\ \isacommand{by}\isamarkupfalse%
\ simp\isanewline
\isacommand{qed}\isamarkupfalse%
%
\endisatagproof
{\isafoldproof}%
%
\isadelimproof
\isanewline
%
\endisadelimproof
\isanewline
\isacommand{lemma}\isamarkupfalse%
\ Max{\isacharunderscore}{\kern0pt}mid{\isacharunderscore}{\kern0pt}sums{\isacharunderscore}{\kern0pt}geq{\isacharunderscore}{\kern0pt}{\isadigit{0}}{\isacharcolon}{\kern0pt}\ {\isachardoublequoteopen}Max\ {\isacharparenleft}{\kern0pt}mid{\isacharunderscore}{\kern0pt}sums\ xs{\isacharparenright}{\kern0pt}\ {\isasymge}\ {\isadigit{0}}{\isachardoublequoteclose}\isanewline
%
\isadelimproof
%
\endisadelimproof
%
\isatagproof
\isacommand{proof}\isamarkupfalse%
{\isacharminus}{\kern0pt}\isanewline
\ \ \isacommand{have}\isamarkupfalse%
\ {\isachardoublequoteopen}{\isadigit{0}}\ {\isasymin}\ mid{\isacharunderscore}{\kern0pt}sums\ xs{\isachardoublequoteclose}\ \isacommand{unfolding}\isamarkupfalse%
\ mid{\isacharunderscore}{\kern0pt}sums{\isacharunderscore}{\kern0pt}def\ mids{\isacharunderscore}{\kern0pt}def\ \isacommand{by}\isamarkupfalse%
\ force\isanewline
\ \ \isacommand{then}\isamarkupfalse%
\ \isacommand{show}\isamarkupfalse%
\ {\isacharquery}{\kern0pt}thesis\ \isacommand{using}\isamarkupfalse%
\ finite{\isacharunderscore}{\kern0pt}mid{\isacharunderscore}{\kern0pt}sums\ Max{\isacharunderscore}{\kern0pt}ge\ \isacommand{by}\isamarkupfalse%
\ simp\isanewline
\isacommand{qed}\isamarkupfalse%
%
\endisatagproof
{\isafoldproof}%
%
\isadelimproof
\isanewline
%
\endisadelimproof
\isanewline
\isacommand{lemma}\isamarkupfalse%
\ kadane{\isacharcolon}{\kern0pt}\ {\isachardoublequoteopen}m\ {\isasymge}\ cur\ {\isasymLongrightarrow}\ m\ {\isasymge}\ {\isadigit{0}}\ {\isasymLongrightarrow}\ kadane\ xs\ cur\ m\ {\isacharequal}{\kern0pt}\ max\ m\ {\isacharparenleft}{\kern0pt}max\ {\isacharparenleft}{\kern0pt}cur\ {\isacharplus}{\kern0pt}\ Max\ {\isacharparenleft}{\kern0pt}front{\isacharunderscore}{\kern0pt}sums\ xs{\isacharparenright}{\kern0pt}{\isacharparenright}{\kern0pt}\ {\isacharparenleft}{\kern0pt}Max\ {\isacharparenleft}{\kern0pt}mid{\isacharunderscore}{\kern0pt}sums\ xs{\isacharparenright}{\kern0pt}{\isacharparenright}{\kern0pt}{\isacharparenright}{\kern0pt}{\isachardoublequoteclose}\isanewline
%
\isadelimproof
%
\endisadelimproof
%
\isatagproof
\isacommand{proof}\isamarkupfalse%
\ {\isacharparenleft}{\kern0pt}induction\ xs\ cur\ m\ rule{\isacharcolon}{\kern0pt}\ kadane{\isachardot}{\kern0pt}induct{\isacharparenright}{\kern0pt}\isanewline
\ \ \isacommand{case}\isamarkupfalse%
\ {\isacharparenleft}{\kern0pt}{\isadigit{1}}\ cur\ m{\isacharparenright}{\kern0pt}\isanewline
\ \ \isacommand{then}\isamarkupfalse%
\ \isacommand{show}\isamarkupfalse%
\ {\isacharquery}{\kern0pt}case\ \isacommand{unfolding}\isamarkupfalse%
\ front{\isacharunderscore}{\kern0pt}sums{\isacharunderscore}{\kern0pt}def\ fronts{\isacharunderscore}{\kern0pt}def\ mid{\isacharunderscore}{\kern0pt}sums{\isacharunderscore}{\kern0pt}def\ mids{\isacharunderscore}{\kern0pt}def\ \isacommand{by}\isamarkupfalse%
\ auto\isanewline
\isacommand{next}\isamarkupfalse%
\isanewline
\ \ \isacommand{case}\isamarkupfalse%
\ {\isacharparenleft}{\kern0pt}{\isadigit{2}}\ x\ xs\ cur\ m{\isacharparenright}{\kern0pt}\isanewline
\ \ \isacommand{then}\isamarkupfalse%
\ \isacommand{show}\isamarkupfalse%
\ {\isacharquery}{\kern0pt}case\isanewline
\ \ \ \ \isacommand{apply}\isamarkupfalse%
\ {\isacharparenleft}{\kern0pt}auto\ simp{\isacharcolon}{\kern0pt}\ max{\isacharunderscore}{\kern0pt}front{\isacharunderscore}{\kern0pt}sum\ max{\isacharunderscore}{\kern0pt}mid{\isacharunderscore}{\kern0pt}sums{\isacharunderscore}{\kern0pt}cons\ Let{\isacharunderscore}{\kern0pt}def{\isacharparenright}{\kern0pt}\isanewline
\ \ \ \ \isacommand{by}\isamarkupfalse%
\ {\isacharparenleft}{\kern0pt}smt\ {\isacharparenleft}{\kern0pt}verit{\isacharcomma}{\kern0pt}\ ccfv{\isacharunderscore}{\kern0pt}threshold{\isacharparenright}{\kern0pt}\ Max{\isacharunderscore}{\kern0pt}front{\isacharunderscore}{\kern0pt}sums{\isacharunderscore}{\kern0pt}geq{\isacharunderscore}{\kern0pt}{\isadigit{0}}\ add{\isacharunderscore}{\kern0pt}assoc\ add{\isacharunderscore}{\kern0pt}{\isadigit{0}}{\isacharunderscore}{\kern0pt}right\ max{\isachardot}{\kern0pt}assoc\ max{\isachardot}{\kern0pt}coboundedI{\isadigit{1}}\ max{\isachardot}{\kern0pt}left{\isacharunderscore}{\kern0pt}commute\ max{\isachardot}{\kern0pt}orderE\ max{\isacharunderscore}{\kern0pt}add{\isacharunderscore}{\kern0pt}distrib{\isacharunderscore}{\kern0pt}left\ max{\isacharunderscore}{\kern0pt}add{\isacharunderscore}{\kern0pt}distrib{\isacharunderscore}{\kern0pt}right{\isacharparenright}{\kern0pt}\isanewline
\isacommand{qed}\isamarkupfalse%
%
\endisatagproof
{\isafoldproof}%
%
\isadelimproof
\isanewline
%
\endisadelimproof
\isanewline
\isacommand{lemma}\isamarkupfalse%
\ Max{\isacharunderscore}{\kern0pt}front{\isacharunderscore}{\kern0pt}sums{\isacharunderscore}{\kern0pt}leq{\isacharunderscore}{\kern0pt}Max{\isacharunderscore}{\kern0pt}mid{\isacharunderscore}{\kern0pt}sums{\isacharcolon}{\kern0pt}\ {\isachardoublequoteopen}Max\ {\isacharparenleft}{\kern0pt}front{\isacharunderscore}{\kern0pt}sums\ xs{\isacharparenright}{\kern0pt}\ {\isasymle}\ Max\ {\isacharparenleft}{\kern0pt}mid{\isacharunderscore}{\kern0pt}sums\ xs{\isacharparenright}{\kern0pt}{\isachardoublequoteclose}\isanewline
%
\isadelimproof
%
\endisadelimproof
%
\isatagproof
\isacommand{proof}\isamarkupfalse%
{\isacharminus}{\kern0pt}\isanewline
\ \ \isacommand{have}\isamarkupfalse%
\ {\isachardoublequoteopen}front{\isacharunderscore}{\kern0pt}sums\ xs\ {\isasymsubseteq}\ mid{\isacharunderscore}{\kern0pt}sums\ xs{\isachardoublequoteclose}\ \isacommand{unfolding}\isamarkupfalse%
\ front{\isacharunderscore}{\kern0pt}sums{\isacharunderscore}{\kern0pt}def\ mid{\isacharunderscore}{\kern0pt}sums{\isacharunderscore}{\kern0pt}def\ \isacommand{using}\isamarkupfalse%
\ fronts{\isacharunderscore}{\kern0pt}mids\ subset{\isacharunderscore}{\kern0pt}iff\ \isacommand{by}\isamarkupfalse%
\ blast\isanewline
\ \ \isacommand{then}\isamarkupfalse%
\ \isacommand{show}\isamarkupfalse%
\ {\isacharquery}{\kern0pt}thesis\ \isacommand{using}\isamarkupfalse%
\ front{\isacharunderscore}{\kern0pt}sums{\isacharunderscore}{\kern0pt}not{\isacharunderscore}{\kern0pt}empty\ finite{\isacharunderscore}{\kern0pt}mid{\isacharunderscore}{\kern0pt}sums\ Max{\isacharunderscore}{\kern0pt}mono\ \isacommand{by}\isamarkupfalse%
\ blast\isanewline
\isacommand{qed}\isamarkupfalse%
%
\endisatagproof
{\isafoldproof}%
%
\isadelimproof
\isanewline
%
\endisadelimproof
\isanewline
\isacommand{lemma}\isamarkupfalse%
\ mss{\isacharunderscore}{\kern0pt}kadane{\isacharunderscore}{\kern0pt}mid{\isacharunderscore}{\kern0pt}sums{\isacharcolon}{\kern0pt}\ {\isachardoublequoteopen}mss{\isacharunderscore}{\kern0pt}kadane\ xs\ {\isacharequal}{\kern0pt}\ Max\ {\isacharparenleft}{\kern0pt}mid{\isacharunderscore}{\kern0pt}sums\ xs{\isacharparenright}{\kern0pt}{\isachardoublequoteclose}\isanewline
%
\isadelimproof
\ \ %
\endisadelimproof
%
\isatagproof
\isacommand{unfolding}\isamarkupfalse%
\ mss{\isacharunderscore}{\kern0pt}kadane{\isacharunderscore}{\kern0pt}def\ \isacommand{using}\isamarkupfalse%
\ kadane\ Max{\isacharunderscore}{\kern0pt}mid{\isacharunderscore}{\kern0pt}sums{\isacharunderscore}{\kern0pt}geq{\isacharunderscore}{\kern0pt}{\isadigit{0}}\ Max{\isacharunderscore}{\kern0pt}front{\isacharunderscore}{\kern0pt}sums{\isacharunderscore}{\kern0pt}leq{\isacharunderscore}{\kern0pt}Max{\isacharunderscore}{\kern0pt}mid{\isacharunderscore}{\kern0pt}sums\ \isacommand{by}\isamarkupfalse%
\ auto%
\endisatagproof
{\isafoldproof}%
%
\isadelimproof
\isanewline
%
\endisadelimproof
\isanewline
\isacommand{theorem}\isamarkupfalse%
\ mss{\isacharunderscore}{\kern0pt}kadane{\isacharcolon}{\kern0pt}\ {\isachardoublequoteopen}mss{\isacharunderscore}{\kern0pt}kadane\ xs\ {\isacharequal}{\kern0pt}\ Max\ {\isacharbraceleft}{\kern0pt}s{\isachardot}{\kern0pt}\ {\isasymexists}as\ bs\ cs{\isachardot}{\kern0pt}\ xs\ {\isacharequal}{\kern0pt}\ as\ {\isacharat}{\kern0pt}\ bs\ {\isacharat}{\kern0pt}\ cs\ {\isasymand}\ s\ {\isacharequal}{\kern0pt}\ sum{\isacharunderscore}{\kern0pt}list\ bs{\isacharbraceright}{\kern0pt}{\isachardoublequoteclose}\isanewline
%
\isadelimproof
\ \ %
\endisadelimproof
%
\isatagproof
\isacommand{using}\isamarkupfalse%
\ mss{\isacharunderscore}{\kern0pt}kadane{\isacharunderscore}{\kern0pt}mid{\isacharunderscore}{\kern0pt}sums\ mid{\isacharunderscore}{\kern0pt}sums\ \isacommand{by}\isamarkupfalse%
\ auto%
\endisatagproof
{\isafoldproof}%
%
\isadelimproof
\isanewline
%
\endisadelimproof
\isanewline
\isacommand{end}\isamarkupfalse%
\isanewline
%
\isadelimtheory
\isanewline
%
\endisadelimtheory
%
\isatagtheory
\isacommand{end}\isamarkupfalse%
%
\endisatagtheory
{\isafoldtheory}%
%
\isadelimtheory
%
\endisadelimtheory
%
\end{isabellebody}%
\endinput
%:%file=Maximum_Segment_Sum.tex%:%
%:%11=1%:%
%:%27=3%:%
%:%28=3%:%
%:%29=4%:%
%:%30=5%:%
%:%39=7%:%
%:%40=8%:%
%:%41=9%:%
%:%42=10%:%
%:%43=11%:%
%:%44=12%:%
%:%45=13%:%
%:%46=14%:%
%:%48=16%:%
%:%49=16%:%
%:%50=17%:%
%:%51=18%:%
%:%52=19%:%
%:%53=20%:%
%:%54=21%:%
%:%55=21%:%
%:%62=22%:%
%:%63=22%:%
%:%68=22%:%
%:%71=23%:%
%:%72=24%:%
%:%73=24%:%
%:%80=25%:%
%:%81=25%:%
%:%95=27%:%
%:%105=29%:%
%:%106=29%:%
%:%107=30%:%
%:%108=31%:%
%:%109=32%:%
%:%110=33%:%
%:%111=33%:%
%:%112=34%:%
%:%113=35%:%
%:%114=36%:%
%:%115=37%:%
%:%116=37%:%
%:%117=38%:%
%:%118=39%:%
%:%119=40%:%
%:%120=40%:%
%:%121=41%:%
%:%122=42%:%
%:%123=42%:%
%:%130=43%:%
%:%131=43%:%
%:%132=44%:%
%:%133=44%:%
%:%134=45%:%
%:%135=45%:%
%:%136=46%:%
%:%137=46%:%
%:%138=46%:%
%:%139=47%:%
%:%140=47%:%
%:%141=47%:%
%:%142=47%:%
%:%143=48%:%
%:%144=48%:%
%:%145=49%:%
%:%146=49%:%
%:%147=49%:%
%:%148=49%:%
%:%149=50%:%
%:%155=50%:%
%:%158=51%:%
%:%159=52%:%
%:%160=52%:%
%:%167=53%:%
%:%168=53%:%
%:%169=54%:%
%:%170=54%:%
%:%171=54%:%
%:%172=54%:%
%:%173=55%:%
%:%174=55%:%
%:%175=55%:%
%:%176=55%:%
%:%177=56%:%
%:%183=56%:%
%:%186=57%:%
%:%187=58%:%
%:%188=58%:%
%:%191=59%:%
%:%195=59%:%
%:%196=59%:%
%:%201=59%:%
%:%204=60%:%
%:%205=61%:%
%:%206=61%:%
%:%209=62%:%
%:%213=62%:%
%:%214=62%:%
%:%215=62%:%
%:%220=62%:%
%:%223=63%:%
%:%224=64%:%
%:%225=64%:%
%:%228=65%:%
%:%232=65%:%
%:%233=65%:%
%:%234=65%:%
%:%235=65%:%
%:%240=65%:%
%:%243=66%:%
%:%244=67%:%
%:%245=67%:%
%:%252=68%:%
%:%253=68%:%
%:%254=69%:%
%:%255=69%:%
%:%260=69%:%
%:%263=70%:%
%:%264=71%:%
%:%265=71%:%
%:%272=72%:%
%:%273=72%:%
%:%278=72%:%
%:%281=73%:%
%:%282=74%:%
%:%283=74%:%
%:%290=75%:%
%:%291=75%:%
%:%292=75%:%
%:%297=75%:%
%:%300=76%:%
%:%301=77%:%
%:%302=77%:%
%:%309=78%:%
%:%310=78%:%
%:%311=78%:%
%:%316=78%:%
%:%319=79%:%
%:%320=80%:%
%:%321=81%:%
%:%322=81%:%
%:%323=82%:%
%:%324=83%:%
%:%325=84%:%
%:%326=84%:%
%:%327=85%:%
%:%328=86%:%
%:%329=86%:%
%:%336=87%:%
%:%337=87%:%
%:%338=87%:%
%:%343=87%:%
%:%346=88%:%
%:%347=89%:%
%:%348=89%:%
%:%355=90%:%
%:%356=90%:%
%:%357=91%:%
%:%358=91%:%
%:%359=91%:%
%:%360=92%:%
%:%361=92%:%
%:%362=92%:%
%:%363=92%:%
%:%364=92%:%
%:%365=93%:%
%:%366=93%:%
%:%367=93%:%
%:%368=93%:%
%:%369=94%:%
%:%370=94%:%
%:%371=94%:%
%:%372=94%:%
%:%373=94%:%
%:%374=95%:%
%:%380=95%:%
%:%383=96%:%
%:%384=97%:%
%:%385=97%:%
%:%392=98%:%
%:%393=98%:%
%:%394=99%:%
%:%395=99%:%
%:%396=100%:%
%:%397=100%:%
%:%398=101%:%
%:%399=101%:%
%:%400=101%:%
%:%401=102%:%
%:%402=102%:%
%:%403=102%:%
%:%404=102%:%
%:%405=102%:%
%:%406=103%:%
%:%407=103%:%
%:%408=103%:%
%:%409=104%:%
%:%410=104%:%
%:%411=105%:%
%:%412=105%:%
%:%413=106%:%
%:%414=106%:%
%:%415=106%:%
%:%416=106%:%
%:%417=107%:%
%:%418=107%:%
%:%419=107%:%
%:%420=107%:%
%:%421=108%:%
%:%422=108%:%
%:%423=109%:%
%:%424=109%:%
%:%425=110%:%
%:%426=110%:%
%:%427=110%:%
%:%428=110%:%
%:%429=110%:%
%:%430=111%:%
%:%431=111%:%
%:%432=111%:%
%:%433=111%:%
%:%434=111%:%
%:%435=112%:%
%:%436=112%:%
%:%437=113%:%
%:%438=113%:%
%:%439=114%:%
%:%440=114%:%
%:%441=114%:%
%:%442=114%:%
%:%443=115%:%
%:%449=115%:%
%:%452=116%:%
%:%453=117%:%
%:%454=117%:%
%:%457=118%:%
%:%461=118%:%
%:%462=118%:%
%:%463=118%:%
%:%468=118%:%
%:%471=119%:%
%:%472=120%:%
%:%473=120%:%
%:%476=121%:%
%:%480=121%:%
%:%481=121%:%
%:%486=121%:%
%:%489=122%:%
%:%490=123%:%
%:%491=123%:%
%:%494=124%:%
%:%498=124%:%
%:%499=124%:%
%:%504=124%:%
%:%507=125%:%
%:%508=126%:%
%:%509=126%:%
%:%512=127%:%
%:%516=127%:%
%:%517=127%:%
%:%518=127%:%
%:%523=127%:%
%:%526=128%:%
%:%527=129%:%
%:%528=129%:%
%:%531=130%:%
%:%535=130%:%
%:%536=130%:%
%:%541=130%:%
%:%544=131%:%
%:%545=132%:%
%:%546=132%:%
%:%549=133%:%
%:%553=133%:%
%:%554=133%:%
%:%559=133%:%
%:%562=134%:%
%:%563=135%:%
%:%564=135%:%
%:%567=136%:%
%:%571=136%:%
%:%572=136%:%
%:%577=136%:%
%:%580=137%:%
%:%581=138%:%
%:%582=138%:%
%:%585=139%:%
%:%589=139%:%
%:%590=139%:%
%:%591=139%:%
%:%605=142%:%
%:%615=144%:%
%:%616=144%:%
%:%617=145%:%
%:%618=146%:%
%:%620=148%:%
%:%621=149%:%
%:%622=150%:%
%:%623=150%:%
%:%624=151%:%
%:%625=152%:%
%:%626=152%:%
%:%633=153%:%
%:%634=153%:%
%:%635=154%:%
%:%636=154%:%
%:%637=154%:%
%:%638=154%:%
%:%639=155%:%
%:%640=155%:%
%:%641=155%:%
%:%642=155%:%
%:%643=155%:%
%:%644=156%:%
%:%645=156%:%
%:%646=156%:%
%:%647=156%:%
%:%648=156%:%
%:%649=157%:%
%:%655=157%:%
%:%658=158%:%
%:%659=159%:%
%:%660=159%:%
%:%667=160%:%
%:%668=160%:%
%:%669=161%:%
%:%670=161%:%
%:%671=161%:%
%:%672=161%:%
%:%673=162%:%
%:%674=162%:%
%:%675=162%:%
%:%676=162%:%
%:%677=162%:%
%:%678=163%:%
%:%684=163%:%
%:%687=164%:%
%:%688=165%:%
%:%689=165%:%
%:%696=166%:%
%:%697=166%:%
%:%698=167%:%
%:%699=167%:%
%:%700=168%:%
%:%701=168%:%
%:%702=168%:%
%:%703=168%:%
%:%704=168%:%
%:%705=169%:%
%:%706=169%:%
%:%707=170%:%
%:%708=170%:%
%:%709=171%:%
%:%710=171%:%
%:%711=171%:%
%:%712=172%:%
%:%713=172%:%
%:%714=173%:%
%:%715=173%:%
%:%716=174%:%
%:%722=174%:%
%:%725=175%:%
%:%726=176%:%
%:%727=176%:%
%:%734=177%:%
%:%735=177%:%
%:%736=178%:%
%:%737=178%:%
%:%738=178%:%
%:%739=178%:%
%:%740=178%:%
%:%741=179%:%
%:%742=179%:%
%:%743=179%:%
%:%744=179%:%
%:%745=179%:%
%:%746=180%:%
%:%752=180%:%
%:%755=181%:%
%:%756=182%:%
%:%757=182%:%
%:%760=183%:%
%:%764=183%:%
%:%765=183%:%
%:%766=183%:%
%:%767=183%:%
%:%772=183%:%
%:%775=184%:%
%:%776=185%:%
%:%777=185%:%
%:%780=186%:%
%:%784=186%:%
%:%785=186%:%
%:%786=186%:%
%:%791=186%:%
%:%794=187%:%
%:%795=188%:%
%:%796=188%:%
%:%799=189%:%
%:%804=190%:%



\bibliographystyle{abbrv}
\bibliography{root}

\end{document}
